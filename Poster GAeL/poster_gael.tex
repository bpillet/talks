\documentclass[a3, 14pt, ruledsections]{sciposter}

\usepackage{multicol}
\usepackage{graphicx}
\usepackage{amsmath, amsthm, amsfonts}
\usepackage[latin1]{inputenc}

%\setlength{\columnsep}{1.6cm}
%\setlength{\textwidth}{24.3cm}
%\setlength{\evensidemargin}{0.8cm}
%\setlength{\oddsidemargin}{0.8cm}

\renewcommand{\titlesize}{\huge}
\renewcommand{\authorsize}{\normalsize}
\renewcommand{\instsize}{\small}


\title{
Hyperk\"{a}hler metrics and application to the twistor space\bigskip
}
\author{Basile \textsc{Pillet}}
\institute{University of Rennes 1\\
            And Rennes Mathematical Institute for Research (IRMAR)}
\email{Advisor: Prof. Christophe \textsc{Mourougane}}

\begin{document}

\conference{\textbf{GAeL XXII}, Trieste, 2014}
\maketitle

%should I write hyperkahler or hyperkähler ?

\begin{multicols}{2}
\section*{The quest for hyperk\"{a}hler metrics}

\PARstart{H}{yperk\"{a}hler} structures can be defined purely in terms of Riemannian geometry. A hyperk\"{a}hler manifold is then a Riemannian manifold having the symplectic group as holonomy group.

\PARstart{O}{ne} of the many consequences of \textsc{Yau}'s theorem is the existence of a unique (Ricci-flat) hyperk\"{a}hler metric in any k\"{a}hler class of a compact irreducible holomorphic symplectic manifold. However there is no explicit expression of such metric.

Actually, very few compact hyperk\"{a}hler manifolds are known, all stemming from \textsc{K3} surfaces. Expressions of hyperk\"{a}hler metrics are even scarcer.

\subsection*{Quiver construction}

\PARstart{N}{akajima} (see \cite{nakajima, Schiffmann}) developed a very general construction of (non-compact) holomorphic symplectic varieties using the \textit{hyperk\"{a}hler quotient} techniques presented in \cite{HKLR}. Such quotient variety can be endowed with any tensor-structure coming from above, assuming the tensor satisfy some invariance properties. Moreover the quotient metric can be explicitly described and would give examples of hyperk\"{a}hler metrics.

For example \textsc{Hitchin} has recovered several known hyperk\"{a}hler metrics, such as \textsc{Calabi-Eguchi-Hanson}'s metric on $T^*\mathbb{P}^n$, using this construction.

\section*{Twistor space}

\PARstart{A}{ny} hyperk\"{a}hler manifold comes with a $\mathbb{P}^1$-family of complex structures. All those data can be gathered into a single manifold : \textit{the twistor space} $Z$. It is fibered over $\mathbb{P}^1$, and any fiber correspond to the given complex structure on $M_\text{diff}$. It happens that the twistor space is a complex manifold and the forenamed fibration is holomorphic.

\[
\begin{array}{ccc}
Z & \rightarrow & M_\text{diff}\\
\left\downarrow\rule{0cm}{1cm}\right. & & \\
\mathbb{P}^1
\end{array}
\]

In the moduli space of hyperk\"{a}hler manifolds the rational curves arising from the twistor construction carry a lot of information \cite{Verbitsky}.

\section*{My research}
\PARstart{T}{hus} my research focusses on finding exact expressions of metrics on hyperk\"{a}hler manifolds. In case the metric has an exact expression, it would certainly be analytic, and therefore one could effectively compute local coordinates on the twistor space using integrability theorems and the expression of the almost complex structure given in \cite{HKLR}.

Such coordinates would allow to unravel many mysteries of this twistor space and would have many applications from algebraic geometry to theoretical physics.

\PARstart{A}{} first example would be to recover the result of \textsc{A. Fujiki} \cite{Fujiki} identifying the twistor space of $T^*\mathbb{P}^1$ with an open subset of \textsc{Nagata}'s threefold (the first known example of non-projective proper smooth algebraic variety).

\subsection*{Other}

\PARstart{A}{mong} my other (but related) research interests, one can cite quaternionic and octonionic geometry and also most of  \textsc{Penrose}'s work : including tiling, his notation for mono{\"{i}}dal categories, twistor theory and also his "mathematical planonism".

\end{multicols}
\bibliographystyle{amsplain}

\bibliography{biblio}

\end{document} 
