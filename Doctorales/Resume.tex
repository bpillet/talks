\documentclass[10pt,a4paper]{article}
\usepackage[utf8x]{inputenc}
\usepackage[francais]{babel}
\usepackage[T1]{fontenc}
\usepackage{amsmath}
\usepackage{amsfonts}
\usepackage{amssymb}
\author{Basile Pillet}
\title{\scshape La transformée de Penrose}
\date{}

\begin{document}
\maketitle

\paragraph{Résumé~:}
La transformée de Penrose permet de de relier le monde (vaste et compliqué) des solutions d'équations différentielles au monde (rigide et spartiate) des fonctions holomorphes. L'idée fondamentale est de voir un point comme une droite (projective complexe) d'un autre espace plus grand.

Pour les physiciens, c'est la révolution : Un point de l'espace à un instant précis est une sphère dans l'\textit{espace des twisteurs} et un simple photon, qui est une solution des equations de Maxwell, peut être considéré comme une fonction méromorphe sur cet espace...

Dans cet exposé, on introduira les outils de géométrie pour pouvoir poser le cadre de la transformée de Penrose.

\nocite{Ward-Wells, TwistorThED}
\bibliographystyle{amsalpha}
\bibliography{/home/basile/Git/Bibliography/full.bib}
\end{document}