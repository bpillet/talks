\documentclass[11pt,draft,makeidx]{article}

\usepackage[]{dipneuste} % Options : ct, thm, couleur, draft, minimal

\pagenumbering{gobble}
\geometry{reset}

\title{\textsc{Twistor Space of an Hyperkähler Manifold}}
\date{December 2nd, 2015}
\author{\name{Basile}{Pillet}\\ {\small (University of Rennes 1)}}

\begin{document}
\selectlanguage{english}
\maketitle
\textit{Hyperkähler} manifolds are riemannian manifolds with $\text{Sp}(n)$ holonomy. To each such, there is an associated complex manifold~: the \textit{twistor space} \cite{HKLR}.

The goal of this talk is to explain what information on the manifold are encrypted whithin the twistor space. For this we will consider two examples~:

First, we will see how a powerfull tool, twistor space is, to deal with \textit{hyperbolicity} (in Brody's sense) for hyperkahler manifolds ; by presenting a result of F. Campana \cite{Campana}.

Then, we will describe the "classical" twistor theory. In this case, the \textit{Penrose transform} yields a marvellous interplay between the riemannian structure (curvature) on one side, and the complex structure on the other side.


\vfill

\bibliographystyle{amsalpha}
\bibliography{/home/basile/Git/Bibliography/full}
\end{document}