\documentclass[11pt,draft,makeidx]{article}

\usepackage[]{dipneuste} % Options : ct, thm, couleur, draft, minimal

\pagenumbering{gobble}
\geometry{reset}
\title{\textsc{Espace des Twisteurs d'une Variété Hyperkählérienne}}
\date{2 décembre 2015}
\author{\name{Basile}{Pillet}\\ {\small (Université de Rennes 1)}}

\begin{document}
\maketitle
Les variétés \textit{hyperkählériennes} sont des variétés riemanniennes ayant une holonomie de type $\text{Sp}(n)$. Chacune peut se voir associer une variété complexe~: son \textit{espace des twisteur} \cite{HKLR}.

Le but de cet exposé est d'expliquer quelles informations sur la variété hyperkählérienne sont encodées dans l'espace des twisteurs. Et ce, sur deux exemples particuliers~:

On verra, premièrement, comment l'espace des twisteurs nous permet d'attaquer la question de l'\textit{hyperbolicité} (au sens de Brody) pour les variétés hyperkählériennes ; en présentant un résultat de F. Campana \cite{Campana}.

Enfin, on présentera aussi le cas "classique" des espaces de twisteurs. Dans ce cas, la \textit{transformée de Penrose} établit une relation miraculeuse entre d'une part la structure riemannienne (courbure) et d'autre part la structure complexe de l'espace des twisteurs.


\vfill

\bibliographystyle{amsalpha}
\bibliography{/home/basile/Git/Bibliography/full}
\end{document}