\documentclass[10pt,a4paper]{article}
\usepackage[utf8]{inputenc}
\usepackage[francais]{babel}
\usepackage[T1]{fontenc}
\usepackage{amsmath}
\usepackage{amsfonts}
\usepackage{amssymb}
\author{Basile Pillet}
\title{Avatars du lemme de \textsc{Yoneda}}
\date{}

\begin{document}
\maketitle
\begin{center}
\textit{AUCUNE connaissance de théorie des catégories n'est requise ;\\AUCUNE connaissance de théorie des catégories ne sera acquise au cours de l'exposé !}
\end{center}
\paragraph{Résumé~:}
Le lemme de \textsc{Yoneda} est peut-être le résultat le plus profond des mathématiques. La majeure partie de l'exposé sera dédié à présenter divers théorèmes élémentaires (parfois triviaux) dans divers domaines des mathématiques : On parlera, sans distinction aucune, de théorie des ensembles, de combinatoire, de logique, d'analyse fonctionnelle, de théorie des groupes, d'algèbre linéaire, de géométrie. Tous ces théorèmes découleront tous du lemme de \textsc{Yoneda}.\\
Fort de ces exemples, on pourra entrapercevoir la philosophie de ce résultat, l'idée derrière certains travaux de \textsc{Grothendieck} et peut-être le sens de la vie.
\end{document}