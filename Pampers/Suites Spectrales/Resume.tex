\documentclass[10pt,a4paper]{article}
\usepackage[utf8]{inputenc}
\usepackage[francais]{babel}
\usepackage[T1]{fontenc}
\usepackage{amsmath}
\usepackage{amsfonts}
\usepackage{amssymb}
\author{Basile Pillet}
\title{Les Suites Spectrales}
\date{}

\begin{document}
\maketitle

\subsubsection*{\textsc{OBI-WAN}~:}
Maître, qu'est-ce qu'une suite spectrale ?
\medskip
\subsubsection*{\textsc{QUI-GON}~:}
Un moyen de calcul… j'espère
\bigskip

\paragraph{}
\hfill------------\hfill
\medskip

\paragraph{} Dans cet exposé, missa vous présente un outil terrific pour calcul cohomologicum. Ilssa y va y avoir des groupas finito et boco boco d'algebra Tek Tek Beurk.


\nocite{Voisin, McCleary}
\bibliographystyle{amsalpha}
\bibliography{/home/basile/Git/Bibliography/full.bib}
\end{document}