\documentclass[10pt,a4paper]{article}
\usepackage[utf8]{inputenc}
\usepackage[francais]{babel}
\usepackage[T1]{fontenc}
\usepackage{amsmath}
\usepackage{amsfonts}
\usepackage{amssymb}
\author{Basile Pillet}
\title{Additionner et diviser : Quelques questions ouvertes d'arithmétique}
\date{}

\begin{document}
\maketitle
\paragraph{Résumé~:}
Un collégien sait additionner et multiplier de nombres. Un mathématicien, ne sait pas faire mieux.

Plus précisément, étant donné deux nombres $a$ et $b$ dont on connaît les diviseurs. On ne sait presque rien dire des diviseurs de $a+b$. Et tellement rien que toutes les plus vieilles questions ouvertes d'arithmétique sont de ce genre (j'en présenterai au moins $3$). 

La difficulté de ces questions a amené Paul Erdös à dire "\textit{Mathematics is not yet ready for such problems}". Inutile de préciser que cet exposé l'est encore moins.
\end{document}