\documentclass[a4paper,draft]{amsart}
\usepackage[thm,couleur]{dipneuste}
\geometry{a4paper, hmargin=7em, vmargin=7em}
%\usepackage{showkeys} %INDIQUE LES REFERENCES ET LABELS

\title[Quotients symplectiques et hyperkählériens]{Quotients symplectiques et hyperkählériens}
%\author{Basile Pillet} 
%\address{Basile Pillet, Université Rennes 1}
%\email{basile.pillet@univ-rennes1.fr}
\date\today

\begin{document}
\section{Un premier quotient symplectique : $\C^2/\!/U(1)$}
\subsection{Structures et notations}
Soit $M = \C^2$ munit des coordonnées $w = (w^1,w^2) = (x^1 + i y^1,x^2 + i y^2)$, de la structure kählérienne : 
\[
g = (\dd x^1)^2 + (\dd y^1)^2 + (\dd x^2)^2 + (\dd y^2)^2
\]
et de la structure complexe
\[
I \dpp{}{w^k} = i \dpp{}{w^k}
\]
qui induisent une structure symplectique réelle
\[
\omega = g(\_,I\_) = \dd w^1 \wedge \dd \bar{w^1} + \dd w^2 \wedge \dd \bar{w^2} = \dd x^1 \wedge \dd y^1 + \dd x^2 \wedge \dd y^2
\]


\bigskip
\begin{center}
\textit{On notera une expression de la forme 
\[
\sum_{k=1,2} f(x^k,y^k)
\]
par $f(x,y)$.}
\end{center}
\bigskip

Ainsi les structures kählériennes et symplectiques s'écrivent
\[
g = (\dd x)^2 + (\dd y)^2 \qquad \omega = \dd x \wedge \dd y
\]

\subsection{L'action du groupe $U(1)$}
Le groupe de lie $G = U(1)$ agit sur $M$ par homothétie
\[
g \cdot w = (gw^1,gw^2)
\]
On remarque que cette action préserve la forme symplectique $\omega$.

A un point $w \in  M$ fixé, on peut associer une application lisse
\[
\left(
\begin{array}{rcl}
G & \longrightarrow & M \\
g & \mapsto & g\cdot w
\end{array}
\right)
\]
Qui s'exprime dans les coordonnées réelles par
\[
e^{i\theta}\cdot (x+ i y) = (\cos(\theta)x -\sin(\theta)y) + i (\cos(\theta)y + \sin(\theta)x)
\]
D'où en $\theta = 0$,
\[
e^{i\theta + it}\cdot (x+ i y) = (\cos(t)x-\sin(t)y) + i (\cos(t)y +\sin(t)x) = x + i y + t(-y+ix) + o(t)
\]
Ainsi la différentielle en $g = 1$ nous donne
\[
\left(
\begin{array}{rcl}
T_1 G & \longrightarrow & T_w M \\
it & \mapsto & -ty\dpp{}{x} + tx\dpp{}{y}
\end{array}
\right)
\]

On notera $X_w^t$ le vecteur ainsi obtenu. En faisant varier $w$, on obtient un champ de vecteur $X^t$ lisse sur $M$, ce champ de vecteur représente l'action infinitésimale de $G$ sur $M$.

\subsection{Application moment}
Considérons $\omega(X^t,\_)$ le produit intérieur de $\omega$ par $X^t$.
\[
\omega(X^t,\_) = -ty\dd y -tx \dd x = -\dfrac{t}{2}\dd (yy + xx) 
\]
Notons $\mu^t : M \longrightarrow \R$ donnée par \[
w=(x,y) \mapsto -\dfrac{t}{2}( xx + yy )\]

Dès lors le couplage $(t,w) \mapsto \mu^t(w)$ nous donne une application $\mu : M \rightarrow \mathfrak{u}(1)^\vee \cong \R$ : 
$w \mapsto -\demi( xx + yy )$.
Enfin cette application commute avec l'action de $G$ (sur $M$ et sur $\mathfrak{u}(1)^\vee$).

\subsection{Sous-variété de moment}
Considérons dès lors $N_a = \mu^{-1}(a)$. Pour $a \neq 0$ c'est une sous-variété (non vide pour $a<0$). De plus l'équivariance de $\mu$ entraine que $N_a$ est stable sous l'action de $G$.

On fixera dans toute la suite un $a=-\demi$. Alors $N_a = \ens{(w^1,w^2)\in M}{|w^1|^2 + |w^2|^2 = -2a=1}$ est $\S^3_{\C^2}$ la sphère unité de $\C^2$.

\subsection{Quotient symplectique}
$G$ agit proprement sans point fixe sur $N_a$, et l'on peut donc considérer la variété quotient $S = N_a/G$.

On sait qu'on peut associer à tout $w \in N_a = \S^3_{\C^2}$ la droite vectorielle de $\C^2$ qu'elle engendre, ce qui nous donne une application $N_a \rightarrow \Pro^1$ qui s'écrit $(w^1,w^2) \mapsto [w^1 : w^2]$. Or la fibre au dessus d'un élément $[z:z'] \in \Pro^1$ est exactement l'ensemble des $(gz,gz')$ pour $g \in U(1)$. C'est la fibration de \textsc{Hopf}.

Le quotient s'identifie donc à $\Pro^1$.

\paragraph{Forme symplectique quotient} Notons $\varphi$ l'application quotient $ \S^3_{\C^2} \rightarrow \Pro^1$. On notera $z=u+iv$ la coordonnée correspondant à $w^1/w^2$. Au point $(w^1,w^2) = (0,1)$ on a
\[
\varphi_* \dpp{}{x^1} = \dpp{}{u} \quad 
\varphi_* \dpp{}{y_1} = \dpp{}{v}
\]
Maintenant par action de $Sl_2$ et $PGl_2$ sur respectivement $\S^3$ et $\Pro^1$, on peut envoyer le point $(0,1)$ et son image $[0:1]$ sur n'importe quel point, en transportant avec eux les vecteurs tangents. En particulier, on peut déterminer des relevés $X$ et $Y$ de $\partial_u$ et $\partial_v$ dans $\S^3$.

Par exemple la matrice
\[
\begin{bmatrix}
1 & z\\
0 & 1
\end{bmatrix}
\]
de $Sl_2(\C)$ envoie $[0:1]$ sur $[z:1]$ dans $\Pro^1$. Si $z=u+iv$ alors son action sur $T\S^3$ est donnée par
\begin{eqnarray*}
\partial_{x^1}, \partial_{y^1} & \mapsto & \partial_{x^1}, \partial_{y^1}\\
\partial_{x^2} & \mapsto & u\partial_{x^1}+v\partial{y^1} + \partial{x^2}\\
\partial_{y^2} & \mapsto & u\partial_{y^1}-v \partial_{x^1}+ \partial{y^2}
\end{eqnarray*}
\section{Un exemple de quotient hyperkählérien : $\C^2 \times \C^2/\!/\!/U(1)$}
\subsection{Structures et notations}
Soit $M = \C^2 \times \C^2$ munit des coordonnées $(w,z)$ où $w = (w^1,w^2) = (x^1 + i y^1, x^2 + i y^2)$ et $z = (z^1,z^2) = (u^1 + i v^1, u^2 + i v^2)$.
On considère sur $M$ les structures complexes suivantes
\[
I = 
\begin{bmatrix}
i1_2 & 0 \\
0 & -i1_2
\end{bmatrix}
\quad
J = 
\begin{bmatrix}
0 & 1_2 \\
-1_2 & 0
\end{bmatrix}
\quad
K = 
\begin{bmatrix}
0 & i1_2 \\
i1_2 & 0
\end{bmatrix}
\]
Ces trois structures complexes vérifient les relations quaternioniques $I² = J² = K² = IJK = -1_4$ et de plus sont orthogonales pour la structure kählérienne
\[
g = (\dd x)^2 + (\dd y)^2 + (\dd u)^2 + (\dd v)^2
\]
Ces données munissent $M$ d'une structure de variété hyperkählérienne.

\subsection{Action du groupe unitaire}
Comme précédemment, le groupe $G = U(1)$ agit sur $M$ par homothétie : $g\cdot m = (g \cdot w, g \cdot z) = (gw^1,gw^2,gz^1,gz^2)$.
Son action en coordonnées réelles s'écrit
\[
e^{i\theta}[(x,y),(u,v)] = \left[(\cos(\theta)x-\sin(\theta)y,\sin(\theta)x+\cos(\theta)y),(\cos(\theta)u-\sin(\theta)v,\sin(\theta)u+\cos(\theta)v)\right]
\]

Comme précédemment on peut calculer les vecteurs tangents qui proviennent de l'algèbre de Lie de $G$ et qui représentent les déplacement infinitésimaux le long d'une orbite.

\[
X_m^t = -ty\dpp{}{x} + tx\dpp{}{y} - tv\dpp{}{u} + tu\dpp{}{v}
\]

\subsection{Structure tri-symplectique}
Les trois formes de Kähler $\omega_I,\omega_J,\omega_K$ associées au structures complexes $I,J$ et $K$ munissent $M$ de $3$ structures symplectiques réelles. De plus l'action du groupe $G$ préserve chacune de ces $2$-formes.
\begin{eqnarray*}
\omega_I & = & -\dd x \wedge \dd y + \dd u \wedge \dd v\\
\omega_J & = & \dd x \wedge \dd u + \dd y \wedge \dd v\\
\omega_K & = & -\dd x \wedge \dd v + \dd y \wedge \dd u
\end{eqnarray*}

En calculant le produit intérieur de des formes symplectiques par les vecteurs provenant de l'algèbre de Lie de $G$, on obtient $3$ applications moment
\begin{eqnarray*}
\mu_I & = & \\
\mu_J & = & \\
\mu_K & = & 
\end{eqnarray*}
\end{document}