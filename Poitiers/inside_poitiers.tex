\chapter{Talk}
\section{Stronger Bend and Break Lemma}
Chap 3.3
p.63
\subsection{Example(s)}
\subsubsection{Example 1}
\[
f_t : \left(
\begin{array}{ccc}
\mathbb{P}^1 & \longrightarrow & \mathbb{P}^2 \\  {}
 [ u : v ] & \mapsto &  [ u^2 : tuv : v^2 ]
\end{array} 
\right)
\]

Étude du cas limite $t=0$.
\subsubsection{Example 2}
Étude du cas limite $t=\infty$.
\subsection{Bend and Break 
with bounds on degree}
Proposition 3.5
\subsubsection{Statement}
\paragraph*{What is $H \cdot C$ ?}
Que signifie $H \cdot C$ ?

Possibilités
\begin{enumerate}
\item $H \cdot f_* C $
\item deg$_C(f^* H)$
\end{enumerate}

On a dans le texte :
$e^*H \cdot \varepsilon^*C = H \cdot C$
\paragraph*{In the smooth case}
\subsubsection{Proof}
\paragraph*{(A) Normalisation}
\paragraph*{"Normalisation of the image"}
L'image $f_*C$ est un 1-cycle
(non réduit en quelque sorte)
On prend $C''$ une componante et on note $C'$ sa normalisé.
\paragraph*{(B) Compactification}
\paragraph*{lemme de rigidité}
\paragraph*{(C) Resolution of singularities}
\paragraph*{$E_{ij} \cdot E_{kl}$}
For $i=1 \cdots b$, we denote by $E_{i1}, E_{i2}, \cdots, E_{in_i}$ the (effective) inverse images on S of the (-1)-exceptional curves that appear every time some point \textit{lying over} $\{c_i\}\times \overline{T}$ is blown up. We have : 
\[
E_{ij} \cdot E_{kl} = -\delta_{ik}\delta_{jl}
\]



$E_{ij} \cdot T_i = 1$ if the blown up point is on the (smooth) strict transform of $\{c_i\} \times \overline{T}$ and $0$ otherwise. 



Dans le cas de résolution de singularités type cusp : on éclate le cusp de $T$ puis on rééclate le point (double) de rencontre en $T$ et E en un nouveau diviseur exceptionnel $F$ …

Donc si c'etait $T = c_i \times T$
On aurait $E_{i1} = E$
$E_{i2} = F$

et $E_{i1} \cdot E_{i2} = E \cdot F = 1$   -> contradiction avec ci-dessus !


Pareil, si la formule tout en haut est fausse, alors comme on montre que $a_{ij} \geq 0$ car avec cette formule on a :
$e^* H \cdot E_{ij} = +a_{ij}$ et on sait que $H$ est nef.
\paragraph*{(D) Decomposition in $N^1(X)_R$}
\paragraph*{$G \cdot T_i = 0$}
\paragraph*{(E) Hodge Index Theorem}
\paragraph*{(F) Conclusion}
\section{Not nef anticanonical}
Chap 3.4
p. 66
\subsection{Theorem 3.6}
\subsubsection{Statement}
\subsubsection{Sketch of proof}
\paragraph*{In finite caracteristic}
\paragraph*{Lemma 3.7
Closeness of evaluation map}
\paragraph*{In caracteristic 0}
\subsection{Generic Nefness}
\subsubsection{Definition}
\subsubsection{Example}
\subsubsection{Theorem 3.10}
\paragraph*{Statement}
\paragraph*{Proof}
